\documentclass[11pt]{beamer}
\usetheme{Madrid}
\usepackage[utf8]{inputenc}
\usepackage{url}


\title{Introduction to Computational Science}
\subtitle{Optimizing Traffic Lights in Urban Street Grids}
\author{Klaas Kliffen, Jan Kramer}
\setbeamertemplate{navigation symbols}{}
\date{January 7, 2016}

\begin{document}
\maketitle

\begin{frame}{Introduction}
%What did we model
\end{frame}

\begin{frame}{Overview}
% Global overview of the simulatar
\begin{itemize}
    \item 9 intersection
    \item Boundaries are wrapped
\end{itemize}
\end{frame}

\begin{frame}{Intersection design}
% What is the intersection class
\end{frame}

\begin{frame}{Statistics gathering}
% What did we collect?
    
\end{frame}

\begin{frame}{Demo}
%Show the demo (image of the program)

\end{frame}


\begin{frame}{Results: pattern}
%What did we find
\begin{itemize}
 \item A pattern emerged with long switch timings on the two sided algorithm
 %TODO: show an image of both states
\end{itemize}
\end{frame}


\begin{frame}{Results: mean car waiting time}
\textbf{Definition:} cars that cannot move to the next field, either to a 
red traffic light or another car. Averaged over all lanes.\\~\\
Measured using 200 cars simulating 5000 steps.

\begin{table}
\centering
\begin{tabular}{l|c|c|c|c|c}
algorithm/switch time & 1 & 4 & 8 & 16 & 32\\
\hline
simple & 1 & 4 & 8 & 16 & 32\\
two sided & 1 & 4 & 8 & 16 & 32\\
\end{tabular}
\end{table}
 
\end{frame}

\begin{frame}{Results: cars moved per intersection per step}
\begin{table}
\centering
\begin{tabular}{l|c|c|c|c|c}
algorithm/switch time & 1 & 4 & 8 & 16 & 32\\
\hline
simple & 1 & 4 & 8 & 16 & 32\\
two sided & 1 & 4 & 8 & 16 & 32\\
\end{tabular}
\end{table}
 
\end{frame}


\begin{frame}{Conclusion}
    
\end{frame}

\begin{frame}{Introduction to Computational Science}
\begin{center}
{\large Optimizing Traffic Lights in Urban Street Grids}\\
\end{center}

\begin{center}
Klaas Kliffen, Jan Kramer
\end{center}

%TODO: put a nice image here (instead of asking for questions)
    
\end{frame}



\end{document}
